%&latex
\documentclass[11pt,english]{article}
\usepackage[T1]{fontenc}
\usepackage[latin9]{inputenc}
\usepackage{geometry}
\geometry{verbose,tmargin=1.25in,bmargin=1.25in,lmargin=1.25in,rmargin=1.25in}
\usepackage{url}
\usepackage{amsmath}
\usepackage{amsthm}
\usepackage{amssymb}
\usepackage{graphicx}
\usepackage{setspace}
\usepackage{wasysym}
\usepackage[authoryear]{natbib}
\usepackage{hyperref}
\onehalfspacing

\makeatletter
%%%%%%%%%%%%%%%%%%%%%%%%%%%%%% Textclass specific LaTeX commands.
  \theoremstyle{plain}
  \newtheorem{prop}{\protect\propositionname}
  \theoremstyle{plain}
% \newtheorem{cor}{\protect\corollaryname}
 \newtheorem{cor}{Corollary}

  \theoremstyle{plain}
  \newtheorem{lem}{\protect\lemmaname}
%%%%%%%%%%%%%%%%%%%%%%%%%%%%%% User specified LaTeX commands.
\usepackage{babel}


% set fonts for nicer pdf view
\IfFileExists{lmodern.sty}{\usepackage{lmodern}}{}


\usepackage{babel}
\providecommand{\lemmaname}{Lemma}
  \providecommand{\propositionname}{Proposition}

\date{August 21, 2017}


\usepackage{babel}
\providecommand{\lemmaname}{Lemma}
  \providecommand{\propositionname}{Proposition}
\usepackage{varioref}

\@ifundefined{showcaptionsetup}{}{%
 \PassOptionsToPackage{caption=false}{subfig}}
\usepackage{subfig}
\makeatother

\usepackage{babel}
  \providecommand{\lemmaname}{Lemma}
  \providecommand{\propositionname}{Proposition}

\begin{document}

%+Title
\title{A simple exposition of present-bias}
\author{Karna Basu \& Jonathan Conning}
\date{\today}
\maketitle
%-Title

%+Abstract
\begin{abstract}
    Simple graphical exposition of intertemporal choice and commitment issues for sophisticated and naive present-biased consumers with exact solutions for the CRRA utility case.
\end{abstract}
%-Abstract

%+Contents
\tableofcontents
%-Contents

\section{Introduction}
 

\section{Model}


\subsection{Full-commitment contracts}

We first establish the benchmark `full-commitment' contracts under
perfect competition and monopoly, respectively. To characterize these
familiar commitment contracts, we assume a bank can \textit{costlessly}
offer Zero-self a multi-period contract that binds the consumer's
latter self(ves) to not renegotiate its terms with the same bank or
other banks. Although we are not making the exact mechanisms explicit
yet the bank's ability to credibly commit to not renegotiate such
a contract must rest on the assumption that the bank has credibly
bonded itself to paying a renegotiation penalty ($\kappa$) in the
event of renegotiation and that this penalty is sufficiently high
to deter the bank. In later sections of the paper we will examine
what happens when $\kappa$ is small. Then exogenously sustained commitment
contracts can no longer be sustained and must be replaced by second-best
self-enforcing renegotiation-proof contracts. We can then delve deeper
into what in practice determines $\kappa$ and how it might be endogenously
determined via choices of bank ownership and governance forms, and
shaped by the nature of the market structure.

\subsubsection{Full-commitment contracts under Competition}

\label{sec-FCC} A consumer with time-inconsistent preferences cannot
trust her later selves to stick to her preferred consumption plans.
In this simple three-period setting Zero's concern is that her later
One-self will try to divert resources earmarked for period 2 consumption
to boost period 1 consumption instead. Like a Stackelberg-leader in
a Cournot game, Zero's strategic saving/borrowing choices are affected
by her anticipation of One's best response. A bank may be able to
act as a strategic partner to Zero by offering contracts with commitment
services to help restrict or otherwise control the consumer's later
self(ves)'s best responses.

With an exclusive full-commitment contract the consumer faces no self-control
problem. Zero chooses a contract that commits her One- and Two- selves
to follow the chosen consumption plan. This contract design problem
is solved as a standard utility maximization problem subject to an
inter-temporal budget constraint (or subject to a financial intermediary's
zero-profit condition). Zero-self chooses contract $C_{0}$ to solve:
\begin{equation}
\max_{C_{0}}U_{0}(C_{0})\label{eq:cobj0}
\end{equation}

\begin{equation}
s.t.\quad\Pi_{0}(C_{0};Y_{0})\geq0\label{eq:BPC0}
\end{equation}
The familiar first-order necessary conditions are: 
\begin{equation}
u'\left(c_{0}\right)=\beta\delta(1+r)u'\left(c_{1}\right)=\beta\delta^{2}(1+r)^{2}\\ u'\left(c_{2}\right)
\end{equation}

An increase or decrease to the term $\delta(1+r),$ which enters each
expression above, essentially `tilts' consumption to be more generally
rising or falling over time as $\delta\gtreqless1/(1+r)$. As this
across-the-board level of tilt will not alter key tradeoffs of interest
(unlike the degree of present-bias $\beta$ parameter which does)
we shall impose the assumption that $\delta=\frac{1}{1+r}$ for the
remainder of the analysis. This is without loss of generality but
will greatly unclutter the math. The simplified first-order conditions
are: 
\begin{equation}
u'\left(c_{0}\right)=\beta u'\left(c_{1}\right)=\beta u'\left(c_{2}\right)\label{eq:FOC_comp}
\end{equation}

Along with a binding budget constraint the first-order conditions
allow us to solve for the optimal competitive `full-commitment' contract
$C_{0}^{F}$, so called because latter period selves are committed
to not change it. Conceptually the equilibrium contract will be found
at the tangency between the highest iso-utility surface just touching
the budget hyper-plane. With CRRA utility, a closed form solution
for $C_{0}$ is easily found which we label the competitive full commitment
contract $C_{0}^{F}$.\footnote{All CRRA derivations and closed-form solutions are in the appendix.}
This contract has the property: 
\[
c_{1}^{F}=c_{2}^{F}=\beta^{\frac{1}{\rho}}c_{0}^{F}
\]
Substituting this into the bank's zero profit constraint
we can solve to find:

\begin{equation}
c_{0}^{F} & =\frac{y}{1+2\beta^{\frac{1}{\rho}}}\label{eq:CF}
\end{equation}
Zero-self wants to indulge her present bias (by tilting consumption
toward herself) and then allocate remaining resources evenly across
the remaining two periods.

This solution can be seen graphically in Figure 1a drawn for a consumer
with $\beta=0.5,$ $\rho=1$ and an intertemporal budget constraint
from endowment income with a present value of income $y=\sum y_{t}=300$.
The figure consists of two panels. The left panel depicts the $\left(c_{1},c_{2}\right)$
contract choice, conditional on $c_{0}=c_{0}^{F}$. The allocation
across periods 1 and 2 is determined by a budget constraint (the line
through $QF'$, satisfying $c_{1}+c_{2}=y-c_{0}$) and an optimal
`division rule' which in this case is $c_{1}=c_{2}$ (the line through
$OF'$). This division rule is attainable because full-commitment,
by definition, allows Zero-self to force One-self to respect the first-order
conditions of the maximization problem (\ref{eq:cobj0}).

\begin{figure}
\vspace*{-5cm}
\subfloat[Full-commitment]{
\includegraphics[scale=0.5]{Figure1a.pdf}
}

\subfloat[Renegotiation-proof with $\kappa=0$]{
\includegraphics[scale=0.5]{Figure1b.pdf}
}

\caption{Full-commitment and renegotiation-proof contracts under competition}

\label{fig:twoquad} 
\end{figure}


The right panel depicts the $\left(c_{0},c_{1}\right)$choice conditional
on the division rule being followed in period 1. The budget constraint
(\ref{eq:budgetplus}) is the line going through $yF$ and the first-order
condition (\ref{eq:FOCc0c1}) is the line going through $0F$.

The optimal contract $C_{0}^{F}$ lies at the intersections $F$ and
$F'$. To illustrate, at these parameters, Zero's preferred contract
is $C_{0}^{F}=(150,75,75)$. Whether the consumer borrows or saves
(or pays down debts) in any given period depends on how this consumption
stream matches her autarky consumption stream. If, for example, the
total income of $300$ arrives evenly across periods as $Y_{0}=(100,100,100)$
then this consumption plan would imply borrowing $c_{0}-y_{0}=50$
in period 0 to be repaid as installments of 25 in each of periods
1 and 2. If the stream had instead been $Y_{0}=(200,50,50)$ the consumer
would be seen as saving 50 in period 0 to raise consumption by 25
in each of periods 1 and 2.\footnote{These parameter values are chosen for expositional purposes. In particular
$\rho=1$ implies that period zero consumption will be the same with
or without self-control but the analysis can be easily adapted to
other cases.}

This approach will be useful for further analyzing situations where
Zero-self loses control over the division rule (i.e. loses commitment).

\subsubsection{Full-commitment contracts under Monopoly}

\label{sec:own}

If there is monopoly instead of competition for banking services in
period 0, the analysis is similar except that now we solve for the
optimum full-commitment contract by maximizing bank profits subject
to a consumer participation constraint. The bank solves: 
\begin{align}
\max_{C_{0}} & \;\Pi_{0}\left(C_{0};Y_{0}\right)\label{eq:monop-obj}\\
s.t. & \;U_{0}\left(C_{0}\right)\geq U_{0}^{A}(Y_{0})\label{eq:CPC0}
\end{align}

The first-order tangency conditions are the same as competitive case
given by expressions \ref{eq:FOC_comp}, which for CRRA utility again
implies $c_{1}=c_{2}=\beta^{\frac{1}{\rho}}c_{0}$. Along with the
fact that Zero's participation constraint must bind at a monopoly
optimum these equations allow us to solve for the contract $C_{0}^{mF}$
and corresponding bank profits $\Pi_{0}\left(C_{0}^{mF};Y_{0}\right)$.
Closed form solutions for the CRRA utility case appear as appendix
equations A\ref{eq:c-mf} and A\ref{eq:pi-mf}, respectively. Conceptually,
the optimum contract will be at the tangency point where the highest
iso-profit plane still touches the iso-utility surface associated
with Zero's reservation utility.

The terms of the optimal monopoly contract, and hence also the level
of bank profits achieved will be dependent on the consumer's autarky
utility $U_{0}^{A}(Y_{0})$. The present value of $C_{0}^{mF}$ rises
and profits fall with $U_{0}^{A}$. Since the monopolist retains the
gains from trade and consumer iso-utility (or indifference) surfaces
do not cross, consumption in each period under monopoly will be strictly
lower than under competition at all $Y_{0}$ except in the extreme
case where autarky utility is already optimal ($U_{0}^{A}=U_{0}^{F}$),
in which case the monopolist will trivially offer the utility maximizing
contract to the consumer and will make zero profits.
 
 
%+Bibliography
\begin{thebibliography}{99}
\bibitem{Label1} ...
\bibitem{Label2} ...
\end{thebibliography}
%-Bibliography

\end{document}


